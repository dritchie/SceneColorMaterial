\section*{Appendix}
\label{sec:appendix}

Here we provide a complete listing of features and precise definitions for some of the color properties used in our model.

\subsection*{Color Properties}

\begin{description}
	
\item[Binary] \hfill
	\begin{description}[leftmargin=*]
	  \item[Chroma Difference] measures the squared fraction of perceptual distance due to the chroma channels: $\frac{\delta a^2+\delta b^2}{\delta a^2+\delta b^2+\delta L^2}$
	\end{description}
	
\end{description}

\subsection*{Features}

\begin{description}

\item[Adjacency Features] \hfill
	\begin{description}[leftmargin=*]
	  \item[Enclosure Strengths] are two values which measure how much one neighbor in the adjacency encloses the other and vice versa. Enclosure strength is defined as the number of pixels of the neighboring segment appearing within a 2-pixel neighborhood outside the segment's boundary, normalized by the area of that neighborhood. Out-of-image pixels are counted as part of the neighborhood area.
	  \item[Unary Segment Features] The adjacency feature set also includes the segment feature vectors of the participating segments. We concatenate the two vectors, putting the one with the smallest $L_2$ norm first to enforce a consistent ordering.
	\end{description}
	
\end{description}