\section*{Appendix}
\label{sec:appendix}

We list more precise definitions for some of the color properties and features used in our model (Section ~\ref{sec:spatialCompat}).

\subsection{Color Properties}
\begin{description}
\item[Unary] \hfill
	\begin{description}
	  \item[Colorfulness] measures saturation of a color in \lab space: $\frac{\sqrt{a^2+b^2}}{\sqrt{a^2+b^2+L^2}}$
	  \item[Name Saliency and Name Counts] measure how uniquely a color is named and the distribution of names given to a color, as described in Heer and Stone ~\shortcite{ColorNamingModels}
	\end{description}
\item[Binary] \hfill
	\begin{description}
	  \item[Chroma Difference] A binary color property measuring the squared fraction of perceptual distance due to the chroma channels $\frac{\delta a^2+\delta b^2}{\delta a^2+\delta b^2+\delta L^2}$
	\end{description}
\end{description}
\subsection{Features}
\begin{description}
\item[Group Features] \hfill
	\begin{description}
	  \item[Relative Size] The total area of the group over the image area
	  \item[Segment Spread] 2D covariance matrix of the group's segment centroids
	  \item[Segment Size Statistics] the min, mean, max, and standard deviation of the groups' relative segment sizes
	  \item[Number of Segments] The fraction of segments in the group out of the total number of segments in the image
	\end{description}
\item[Segment Features] \hfill
	\begin{description}
	  \item[Relative Size] The total area of the segment over the image area
	  \item[Normalized Discrete Compactness] measures how compactness based on the relationship between the segment's boundary edges and its area ~\cite{NormalizedDiscreteCompactness}
	  \item[Elongation] measures the relative narrowness of a segment according to the width and height of its minimum area bounding box: $1-\frac{boxWidth}{boxHeight}$. A square is the least elongated.
	  \item[Label] A set of three binary values -- noise, background, and foreground. The noise variable indicates if the segment is a `noise' segment composed of small connected components. The background variable indicates if a segment belongs to a group with the largest connected component and is not `noise'. All other segments belong to the foreground segment.
	  \item[Radial Distance] measures Euclidean distance from the centroid to the center of the image
	\end{description}

\item[Adjacency Features] \hfill
	\begin{description}
	  \item[Enclosure Strengths] are two values which measure how much one neighbor in the adjacency encloses the other and vice versa. Here, enclosure strength of a segment due to a neighbor is defined as the number of pixels of the neighboring segment appearing within a 2-pixel neighborhood outside the segment's boundary, normalized by the area of that neighborhood. Out-of-image pixels are counted as part of the neighborhood area.
	  \item[Unary Segment Features] The other two adjacency features are the segment feature vectors of the participating segments. For the adjacency feature vector, we put the segment feature vector with the smallest $L_2$ norm first to keep a consistent ordering.
	\end{description}
\end{description}