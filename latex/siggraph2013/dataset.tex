\section{Dataset}
\label{sec:dataset}

To build a data-driven model, we need a source of training data. For the experiments in this paper, we used a dataset of colored patterns and their respective source palettes scraped from COLOURlovers, an online community centered around creating and sharing color designs. Artists and enthusiasts can create colored patterns by either creating a template from scratch or coloring an existing template. 

The scraped dataset contains 100 colored patterns for each of 82 artists---8200 colored patterns in total spread over 2908 unique pattern templates. We chose artists in order of their most popular pattern and if he or she had created at least 100 patterns total. The supplemental materials contain the complete list of patterns in our dataset, including links back to the original source on the COLOURlovers website.

The patterns were originally created as vector images, where each region is mapped to a color in the source palette. However, our dataset only contains the rasterized patterns, which is the only format freely available on the website. Thus, we pre-process the rasterized patterns in order to create pattern templates for training. To create a pattern template from a rasterized pattern, we must map each pixel to a color in the source palette. To reduce quantization noise, we look at the closest palette colors in the 8x8 neighborhood for each pixel, and map that pixel to the mode color. However, the quantized image often still has pixel noise, particularly for more complex patterns with layer blending effects. Thus, we group all connected components that are under a threshold size (in this case, 0.05\% image size) under one `noise' segment. Each other pattern segment is defined as a set of connected components that are 2 pixels or closer to each other, to account for noise. Finally, we create color groups from segments with the same palette color.