We present a method for automatically coloring 2D patterns. Our method uses a probabilistic factor graph model that can be sampled to generate new coloring suggestions. The model is trained on input example patterns to statistically capture the desirable stylistic properties of the inputs. Using Markov Chain Monte Carlo, the model can be sampled to generate a diverse set of attractive new colorings for a target pattern. This probabilistic framework is very general and allows for users to guide the generated suggestions via conditional inference or additional soft constraints. We demonstrate results on a variety of coloring tasks, and we evaluate the model through a perceptual study in which participants judged automatic colorings to be as desirable as hand-created ones.


%We present a method for synthesizing 3D object arrangements from examples. Given a few user-provided examples, our system can synthesize a diverse set of plausible new scenes by learning from a larger scene database. We rely on three novel contributions. First, we introduce a \emph{probabilistic model for scenes} based on Bayesian networks and Gaussian mixtures that can be trained from a small number of input examples. Second, we develop a clustering algorithm that groups objects occurring in a database of scenes according to their local scene neighborhoods. These \emph{contextual categories} allow the synthesis process to treat a wider variety of objects as interchangeable. Third, we train our probabilistic model on a mix of user-provided examples and relevant scenes retrieved from the database. This \emph{mixed model} learning process can be controlled to introduce additional variety into the synthesized scenes. We evaluate our algorithm through qualitative results and a perceptual study in which participants judged synthesized scenes to be highly plausible, as compared to hand-created scenes.