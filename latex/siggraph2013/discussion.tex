\section{Discussion and Future Work}
\label{sec:discussion}

Summarize, restate contributions.

Discuss...stuff.
Some suggestions:
\begin{itemize}
\item{How many training images are needed? (though maybe this should be part of the results section. We could look at the change in weight/coefficient differences as the number of training examples increases. Or possibly the fluctation in model score as number of training images increases?)}
\item{What the model does well and what doesn't it do well? (i.e. maybe it does better on certain types of patterns?) }
\end{itemize}


Some ideas for future work:
\begin{itemize}
 \item{Making it real-time: GPU implementation of parallel tempering (cite Merrell)}
 \item{Making it real-time: Hamiltonian MCMC via automatic differentiation (cite Wingate)}
 \item{Beyond a fixed number of color groups: Transdimensional inference.}
  \item{Working with vector artwork, instead of bitmaps...They are much cleaner. Vector/layered artwork is also interesting due to possible blending effects (like mentioned below). Though one good thing about using bitmaps is that they are more readily available.}
 \item{More sophisticated templates and/or model: `Soft' segments with `soft' membership in color groups? Useful for color blending, dealing with more sophisticated artwork.}
 \item{Templatizing images: In line with the above..automatically or semi-automatically generating pattern templates from images. In our training set, we're provided the source color palette, but for other images, we don't necessarily have that. Currently, we could extract a color theme from the image using K-means or some other technique, and then do a quantization, but this is not very sophisticated and doesn't work well for complicated patterns}
 \item{Semantics: use the web to associate color distributions with labels, so that a user can label a region as 'skin' and we can respect that}
\end{itemize}