%%% The ``\documentclass'' command has one parameter, based on the kind of
%%% document you are preparing.
%%%
%%% [annual] - Technical paper accepted for presentation at the ACM SIGGRAPH 
%%%   or SIGGRAPH Asia annual conference.
%%% [sponsored] - Short or full-length technical paper accepted for 
%%%   presentation at an event sponsored by ACM SIGGRAPH
%%%   (but not the annual conference Technical Papers program).
%%% [abstract] - A one-page abstract of your accepted content
%%%   (Technical Sketches, Posters, Emerging Technologies, etc.). 
%%%   Content greater than one page in length should use the "[sponsored]"
%%%   parameter.
%%% [preprint] - A preprint version of your final content.
%%% [review] - A technical paper submitted for review. Includes line
%%%   numbers and anonymization of author and affiliation information.

\documentclass[review]{acmsiggraph}

%%% If you are submitting your paper to one of our annual conferences - the 
%%% ACM SIGGRAPH conference held in North America, or the SIGGRAPH Asia 
%%% conference held in Southeast Asia - there are several commands you should 
%%% consider using in the preparation of your document.

%%% 1. ``\TOGonlineID''
%%% When you submit your paper for review, please use the ``\TOGonlineID''
%%% command to include the online ID value assigned to your paper by the
%%% submission management system. Replace '45678' with the value you were
%%% assigned.

\TOGonlineid{45678}

%%% 2. ``\TOGvolume'' and ``\TOGnumber''
%%% If you are preparing a preprint of your accepted paper, and your paper
%%% will be published in an issue of the ACM ``Transactions on Graphics''
%%% journal, replace the ``0'' values in the commands below with the correct
%%% volume and number values for that issue - you'll get them before your
%%% final paper is due.

%\TOGvolume{0}
%\TOGnumber{0}

%%% 3. ``TOGarticleDOI''
%%% The ``TOGarticleDOI'' command accepts the DOI information provided to you
%%% during production, and which makes up the URLs which identifies the ACM
%%% article page and direct PDF link in the ACM Digital Library.
%%% Replace ``1111111.2222222'' with the values you are given.

%\TOGarticleDOI{1111111.2222222}

%%% 4. ``\TOGprojectURL'', ``\TOGvideoURL'', ``\TOGdataURL'', ``\TOGcodeURL''
%%% If you would like to include links to personal repositories for auxiliary
%%% material related your research contribution, you may use one or more of
%%% these commands to define an appropriate URL. The ``\TOGlinkslist'' command
%%% found just before the first section of your document will add hyperlinked
%%% icons to your document, in addition to hyperlinked icons which point to
%%% the ACM Digital Library article page and the ACM Digital Library-held PDF.

%\TOGprojectURL{http://graphics.stanford.edu/projectNameHere}
%\TOGvideoURL{}
%\TOGdataURL{}
%\TOGcodeURL{}

\title{Amazing Paper Title that Grabs Your Attention}

%%% The ``\author{}'' command takes the names and affiliations of each of the
%%% authors of your paper or abstract. The ``\thanks{}'' command takes the
%%% contact information for each author.
%%% For multiple authors, separate each author's information by the ``\and''
%%% command.

\author{Matthew Fisher\thanks{e-mail: \{mdfisher, dritchie, msavva, hanrahan\}@stanford.edu}\\ Stanford University %
\and Daniel Ritchie\footnotemark[1]\\ Stanford University %
\and Manolis Savva\footnotemark[1]\\ Stanford University %
\and Thomas Funkhouser\thanks{e-mail: funk@cs.princeton.edu}\\ Princeton University %
\and Pat Hanrahan\footnotemark[1]\\ Stanford University}

%%% The ``pdfauthor'' command accepts the authors of the work,
%%% comma-delimited, and adds this information to the PDF metadata.

\pdfauthor{Matthew Fisher, Manolis Savva, Daniel Ritchie, Thomas Funkhouser, Pat Hanrahan}

%% Insert paper keywords here!
\keywords{}

\usepackage{amsmath}
\usepackage{amsfonts}
\usepackage{dsfont}

\usepackage{booktabs}
\usepackage{multirow}

\newcommand{\unit}[1]{\ensuremath{\,\mathrm{#1}}}

\newcommand{\remark}[1]{\textcolor{red}{[#1]}}
%\newcommand{\remark}[1]{}

\begin{document}

%%% A ``teaser'' image appears under the title and affiliation information,
%%% horizontally centered, and above the two columns of text. This is OPTIONAL.
%%% If you choose to have a ``teaser'' image, it needs to be placed between
%%% ``\begin{document}'' and ``\maketitle.''

\teaser{

\centering
%\includegraphics[width=\linewidth]{figs/teaser}

%
%\begin{table}
    \begin{tabular}{ccc}
        \includegraphics[width=.18\linewidth]{figs/teaser00} & \includegraphics[width=.45\linewidth]{figs/teaser01} & \includegraphics[width=.27\linewidth]{figs/teaser02}\vspace{0.6em}\\
        
        \includegraphics[width=.18\linewidth]{figs/teaser10} & \includegraphics[width=.45\linewidth]{figs/teaser11} & \includegraphics[width=.27\linewidth]{figs/teaser12} \\ 
        Input Pattern                   & Sampled Colorings               & Random Colorings                \\
    \end{tabular}
%\end{table}
\caption{This is the awesome caption for the teaser! \remark{Other ideas: show no comparison, show O'Donovan comparison} \remark{Visio PDF export for these images is pretty low quality, so I'd vote for keeping everything in TeX and using PNGs/tables.}}
\label{fig:teaser}
\vspace{1.0em}

}

\maketitle

\begin{abstract}
We present a probabilistic factor graph model for automatically coloring 2D patterns. The model is trained on example patterns to statistically capture their desirable stylistic properties. It incorporates terms for enforcing both color compatibility and spatial arrangements of colors that are consistent with the training examples. Using Markov Chain Monte Carlo, the model can be sampled to generate a diverse set of attractive new colorings for a target pattern. This general probabilistic framework allows users to guide the generated suggestions via conditional inference or additional soft constraints. We demonstrate results on a variety of coloring tasks, and we evaluate the model through a perceptual study in which participants judged automatic colorings to be as desirable as hand-created ones.
\end{abstract}

%%% ACM Computing Review (CR) categories.
%%% See <http://www.acm.org/class/1998/> for details.
%%% The ``\CRcat'' command takes four arguments.

%\begin{CRcatlist}
%  \CRcat{I.3.5}{Computing Methodologies}{Computer Graphics}{Computational Geometry and Object Modeling};
%\end{CRcatlist}

\keywordlist
\copyrightspace

\TOGlinkslist

\section{Introduction}
\label{sec:introduction}

%% Hook and motivation
From graphic and web design, to fashion and fabrics, to interior design, colored patterns are everywhere.~\remark{Point to some example(s)?} While almost anyone with normal color vision can distinguish patterns that they like from those they don't, \emph{creating} attractive patterns is difficult: it requires an extensive working knowledge of color and spatial aesthetics. For the purposes of this paper, we define a colored pattern to have two parts: a \emph{pattern template}, which is a creative decomposition of space into regions, and a set of \emph{colors} assigned to those regions~\remark{Need to get the `by-numbers' part in here, else we're setting the reader's expectations too high}. Even the more constrained task of coloring an existing pattern template can be challenging, as it still exhbis a daunting array of coloring options. Experienced artists and enthusiasts often seek feedback and inspiration from online communities such as Colourlovers\footnote{http://www.colourlovers.com/}.

%% Goals
Wouldn't it be cool to have a tool that could make this process easier by suggesting colorings? Such a tool should generate good-looking patterns--patterns that look like an artist might have made them. It should generate a variety of different suggestions, so that an uncertain or inexperienced user can explore the space of possibilities. It should be able to generate suggestions automatically, but it should also expose controls for users to refine their criteria.

%% Challenges
Building such a tool isn't easy, as it requires one to somehow computationally encode the properties that make great colorings great. There are many principles of aesthetics one might think to use, such as the myriad of different color harmony rules (cite), or principles such as balance, contrast, etc. (cite) But do all of these principles apply to all patterns? When they do, which are most important? Even assuming an answer to these questions, one still has to figure out how to actually \emph{generate} many, diverse new colorings that satisfy the desired properties.

%% Our approach
In this paper, we present a probabilistic approach to automatic pattern colorization. Our main contribution is a probabilistic factor graph model that can be trained on example patterns and sampled to generate new colorings for a target pattern template. The general-purpose factor graph framework allows us to incorporate terms in our model both for color compatibility and spatial stuff. The individual terms, as well as the relative importance of each term, are automatically trained using machine learning techniques, statistically capturing the desirable properties of the example patterns. By combining Markov Chain Monte Carlo sampling with sample-diversification techniques commonly used for information retrieval, our model can generate a wide variety of attractive colorings. And, via the use of conditional probabilistic inference or the addition of simple constraint factors, users can exercise a high degree of control over the generated suggestions.

%% Summary of results
We demonstrate the effectiveness of our model for a variety of constrained and unconstrained coloring suggestion tasks. We also show real-world applications of our automatic coloring system to 3D scene design and web design. Finally, we evaluate the quality of colorings generated by our model through a judgment study with people recruited online. Automatically-colored patterns were significantly preferred to random colorings and indistinguishable from colorings made by an artist.
\section{Background}
\label{sec:background}

Related work has tackled problems similar to the one we address in this paper. While our approach builds on some of the ideas and algorithms presented in past research, none of these methods alone are sufficient to meet the goals of this paper.

%\paragraph{Segmented image colorization}
\paragraph{Creative color support tools}
In the work most similar in goal to our own, Sauvaget and colleagues describe a system that takes as input an image divided into segments and a set of colors and computes a `harmonious' assignment of colors to segments~\shortcite{ColorizationUsingHarmony}. This system can produce pleasing results, but it is based on preset `contrast of proportion' rules and can only select from a fixed palette of colors. User input is limited to specifying the set of colors to be used and their relative amounts in advance. Such requirements stand in conflict with our goal of supporting exploratory coloring, in which the user does not know in advance the colors she wants and would like to see many plausible suggestions. In contrast, our model is trained from artist-created images. It does not require the user to specify any colors in advance, but its probabilistic inference framework is general enough to support this input if the user desires it.

In the theme of user exploration, Meier et al. develop a suite of interactive interfaces with a focus on helping users browse color themes and experiment with compositions~\shortcite{ColorPaletteTools}. However, when creating coloring suggestions for a target image, there is little computational guidance. Suggestions are created by showing random assignments of colors from an appealing theme to image regions. In this paper, we explore generating more intelligent coloring suggestions while taking into account user input.

\paragraph{Color harmony \& compatibility}
%The computer graphics, aesthetics, and psychology literatures are replete with theories and models that try to capture the characteristics that make two or more colors compatible with one another~\cite{CohenOrHarmonization,Munsell,PalmerColorPreference}.
Computer graphics, aesthetics, and psychology literatures introduce many potential theories and models to explain and predict the compatibility of two or more colors~\cite{CohenOrHarmonization,Munsell,PalmerColorPreference}. Recently, O'Donovan and colleagues presented a data-driven model that predicts the numeric ratings people would give to five-color `color themes'~\shortcite{ODonovan}. This model is highly generalizable, and they demonstrate a variety of applications for it, including improving existing themes and suggesting colors to complete a partial theme. Our model includes this compatibility model as one of its terms. However, as we demonstrate in Section~\ref{sec:approach}, this term alone is insufficient to produce good colorings, as it does not take into account the spatial role each color plays in the pattern. Thus, our model uses the distinguishing features of each pattern region and the relationships between regions to constrain the space of colorings that it will suggest.

\paragraph{Natural image colorization}
While there is little work directly addressing pattern colorization, researchers have developed many approaches to colorizing grayscale photographic images. By and large, these methods rely either on direct color input from the user~\cite{ScribbleColorization} or from a `source image' whose color properties should be matched~\cite{TransferColorization}. However, there is one approach that works more automatically by using a set of example images as training data~\cite{MultimodalColorization}. This approach builds up multimodal distributions of color variability given local texture descriptors and combines these into a conditional random field with unary and binary energy terms. The minimum-energy configuration is then found using a graph cut algorithm. Our model employs similar local color distributions but bases them on features from variable-sized pattern regions, rather than small texture patches. We also cast the task of finding good colorizations as a probabilistic inference problem, rather than an optimization. This framework allows our algorithm to explore the space of colorings and suggest multiple good alternatives, whereas a graph cut approach will yield only a single global optimum.

\paragraph{3D model colorization}
Other recent work has tackled automatic coloring of 3D objects. The DressUp! system suggests plausible outfits---including colors for each clothing item---for virtual characters~\cite{DressUp}. It also employs the color compatibility function of O'Donovan et al.~\shortcite{ODonovan} as part of a probabilistic model, but it does not consider the spatiality of color beyond a top-to-bottom ordering. The Material Memex system models the context-dependent correlation between geometric shape and material properties of 3D object parts~\cite{MaterialMemex}. In a related fashion, the system presented in this paper models the context-dependent correlation between pattern regions and color properties of those regions. The Material Memex does not include any explicit consideration of color compatibility. Additionally, its use of multinomial factors requires a quantization of material configuration space. Since the Memex does not consider colors separately from materials, this quantization can greatly restrict the set of colors the model can possibly suggest. In contrast, our model uses continuous probability distributions.
\section{Approach}
\label{sec:approach}

We seek to facilitate the creative coloring process by automatically suggesting pattern colorings. In doing so, we must keep in mind that users may or may not have a target coloring style in mind. In addition, aesthetic taste varies across users and can depend on the situation. Thus, an effective color support system should both output a variety of appealing colorings as well as provide controls for personalizing suggestions to a preferred coloring style.

Our system takes as input a pattern template and outputs suggested colorings for that template. A pattern template specifies which \emph{segments}, or connected components, in an image can be colored in and which segments must map to the same color. For example, an image of a flower on a background may have a template that specifies all petals of the flower must be the same color, and all background segments must be the same color. We refer to the set of segments that map to the same color as a \emph{color group}. Figure~\ref{fig:teaser} shows an example of a pattern template visualized in grayscale, where each lightness level identifies a different color group. This template representation is relatively easy to author from images composed of segments, such as web designs, 3D renderings, and line drawings.

To generate attractive pattern colorings, a reasonable first step is to enforce that colors are by some definition `compatible' with one another. Figure~\ref{fig:ColorCompatOnly} shows several patterns whose colors receive a high score under the color compatibility model of O'Donovan et al.~\shortcite{ODonovan}. While these high-scoring colorings use attractive colors and exhibit a great degree of diversity, they also display several problems. Some background regions may be oversaturated, coming across as too `loud.' Several foreground regions have insufficient contrast with the pattern background, causing them to blend uncomfortably into the background.

\begin{figure}[htb]
\centering
%\includegraphics[width=\columnwidth]{figs/colorCompatOnly.png}
\begin{tabular}{cccc}
\includegraphics[width=.2\columnwidth]{figs/colorCompat/r_0_0_3-75}&
\includegraphics[width=.2\columnwidth]{figs/colorCompat/r_0_1_3-32}&
\includegraphics[width=.2\columnwidth]{figs/colorCompat/r_0_2_3-67}&
\includegraphics[width=.2\columnwidth]{figs/colorCompat/r_0_3_3-70}\\
3.75&3.32&3.67&3.70\vspace{0.5em}\\
\includegraphics[width=.2\columnwidth]{figs/colorCompat/r_1_0_3-74}&
\includegraphics[width=.2\columnwidth]{figs/colorCompat/r_1_1_3-42}&
\includegraphics[width=.2\columnwidth]{figs/colorCompat/r_1_2_3-66}&
\includegraphics[width=.2\columnwidth]{figs/colorCompat/r_1_3_3-39}\\
3.74&3.42&3.66&3.39\\
\end{tabular}
\caption{Patterns whose colors receive high scores under the color compatibility model of O'Donovan et al.~\shortcite{ODonovan}. The score is shown beneath each pattern; a typical pattern scores between 2 and 4. Many results exhibit problems such as adjacent equi-luminant regions and excessively saturated backgrounds.}
\vspace{-1.0em}
\label{fig:ColorCompatOnly}
\end{figure}

To overcome these problems, we turn to examples of well-colored patterns. If we inspect color groups that are large, highly connected, and spread across the entire pattern---indicative features of a background---we can see how saturated they are. This knowledge can prevent us from using excessively `loud' background colors. If we examine the contrast between adjacent pattern regions, we might find that large foreground regions have high contrast with the background but less contrast with thin borders. Enforcing these same properties in our own colorings should lead to better results.

Consequently, the approach we take to pattern coloring is \emph{data-driven}: given a dataset of example patterns, we learn distributions over color properties such as saturation, lightness, and contrast for individual regions and for adjacent regions. We predict these distributions using discriminative spatial features of the pattern, such as the size and shape of different regions. Finally, we use the predicted distributions to score the goodness of pattern colorings.

In the next sections, we introduce the dataset of patterns used for our experiments (Section~\ref{sec:dataset}). Next, we describe the \emph{unary color functions} that we use to score the colors of indvidual pattern regions (Section~\ref{sec:unary}), as well the \emph{pairwise color functions} that score the colors of adjacent regions (Section~\ref{sec:binary}). While color compatibility alone does not predict good pattern colorings, it helps enforce global consistency between colors, and our approach makes use of this ability (Section~\ref{sec:colorCompat}).

We then show how these three types of scoring functions---\emph{unary}, \emph{pairwise}, and \emph{global}---can be combined into one unified model using the framework of probabilistic factor graphs (Section~\ref{sec:model}). The resulting model is very flexible: we can sample from it to generate a variety of new coloring suggestions, train it on different example sets to capture different coloring styles, and add additional constraints to it to support different usage scenarios (Section~\ref{sec:results}).

\section{Probabilistic Model}
\label{sec:model}

We formulate the problem of generating colorings for a pattern template as finding high-probability colorings under a probabilistic model. Formally, we define a pattern template as $\pattern = (\groups, \segments, \colorVars, \features)$, where $\groups$ is a set of individual groups $\group$, $\segments$ is a set of individual segments $\segment$, $\colorVars$ is a set of color variables, and $\features$ is a set of features over different groups, segments, and segment adjacencies within the pattern. Each group $\group$ and segment $\segment$ is associated with a color variable ($\colorVars_\group$ and $\colorVars_\segment$, respectively). All segments within a color group have the same color by definition, so $\colorVars_{\text{group}(\segment)} = \colorVars_\segment$. A coloring $\colors$ is an assignment of colors to color variables.

We define the probability of a coloring for a particular pattern template as a log-linear model:  
\begin{equation*}
 p(\colors | \pattern : \weights) = \frac{1}{Z_\pattern(\weights)} \prod_{\term \in \model} \exp(\termWeight \cdot \termStats(\colors, \pattern))
\end{equation*}
The model $\model$ is comprised of a number of different terms $\term$, where each term scores the goodness of a color assignment based on a term-dependent statistic $\termStats(\colors,\pattern)$. $Z_\pattern(\weights)$ is the pattern-dependent partition function that normalizes the distribution. Each term also has a weight $\termWeight$ which determines its relative contribution to the model; the method used for setting these weights is detailed in Section~\ref{sec:weights}. Bolded symbols are vector-valued while non-bolded symbols are scalar.

This type of model is well-suited to the coloring-generation problem. It is very flexible; in principle, the individual term statistics $\termStats$ can be any real-valued function. In our model, we include terms both for color compatibility as well as for spatial properties defined over groups, segments, and segment adjacencies. Users can also guide the model via additional soft constraint terms (Section~\ref{sec:results}). In addition, it is very easy to compare the relative importance of each term by comparing their weights, which we will do in Section~\ref{sec:weights} to gain some insight into which terms contribute the most to producing attractive colorings. 

The model can also be interpreted graphically as a factor graph (Yeh et al.~\shortcite{YiTingTiledPatterns} provides an accessible introduction for the graphics community). Each type of term contributes different factors $\factor$ to the graph. For example, our color compatibility term contributes a factor over at most five color variables that scores the compatibility of those colors. The spatial terms contribute a unary factor for each color variable and binary factors for color variables that are adjacent in the pattern. Figure~\ref{fig:FactorGraph} shows an example of such a factor graph. Circles denote the color variables, while squares denote the different factors $\factor$, and edges connect each factor to the variables within their scope. 


\begin{figure}[ht]
\begin{tabular}{cc}
\raisebox{4em}{\includegraphics[width=.2\columnwidth]{figs/factorGraphPattern}} &
\includegraphics[width=.7\columnwidth]{figs/factorGraph} \\
\end{tabular} 
\caption{A factor graph for an example pattern template. Factors correspond with the color compatibility and spatial terms from our model. In the figure, nodes are colored according to their corresponding color group in the pattern\remark{Rough figure.}}
\label{fig:FactorGraph}
\end{figure}


\subsection{Sampling}
\label{sec:sampling}

To generate good coloring suggestions, we must sample high-probability states from the model. Rather than sample directly from the distribution encoded by the model, we sample from an \emph{annealed} distribution of the form $p(\colors | \pattern : \weights)^\frac{1}{t}$, where $t$ is a `temperature' term that controls the peakiness of the distribution.
We use the Metropolis-Hastings algorithm (MH), a variant of Markov Chain Monte Carlo (MCMC), to sample coloring suggestions~\cite{Metropolis,Hastings}. MH explores the coloring state space by \emph{proposing} candidate new states; these proposals are accepted with probability proportional to their model score. Our sampler uses the following proposals:
\begin{itemize}
	\item{\textbf{Perturb} a randomly chosen color by $v \sim \mathcal{N}(0, \sigma)$ in RGB color space}
	\item{\textbf{Swap} two randomly chosen colors}
\end{itemize}
where $\sigma$ varies linearly with the model temperature $t$. The sampler chooses between these two proposals with probability $\rho$, which also varies linearly with temperature. Since the RGB color space is bounded, the perturbation proposal draws from a truncated normal distribution in order to maintain ergodicity of the chain~\cite{TruncatedGaussians}.

Asymptotically, MCMC samples states with a frequency proportional to their probability under the model. In practice, it can take a prohibitively long time to explore all the modes of a distribution as complex as the one encoded by our model. We would like our sampler to explore as many modes as possible so that it can suggest multiple, high-probability coloring states.

To accelerate sampling, we use parallel tempering, a technique that runs multiple MCMC chains in parallel at different temperatures and swaps their states periodically~\cite{ParallelTempering}. Large values of $t$ yield flatter probability landscapes, so these `hot' chains are more likely to take large jumps across the state space. `Cool' chains, on the other hand, reject almost all proposed states that do not lead to higher probabilities, thus behaving like local hill-climbing optimizers. Running multiple chains in parallel allows the total system to alternatively explore and refine different coloring configurations.

Finally, we use maximimum marginal relevance (MMR) to enforce diversity in the set of suggestions returned by the sampler~\cite{MMR}. MMR is a technique from information retrieval that re-ranks every item in a list according to a linear combination of relevance (model score, in our case) and similarity to the items preceding it. The similarity metric we use for two colorings $\colors$ and $\tilde{\colors}$ of a pattern $- \sum_{\group \in \groups} {\size_\group \cdot ||\colors_\group - \tilde{\colors}_\group||}$, which is the area-weighted sum of \lab distances between the corresponding colors in each coloring.

\subsection{Weight Learning}
\label{sec:weights}

The model defined thus far has several terms $\term$, and each has a weight parameter $\weights_\term$. These weights control the relative importance of the different terms in the model---so how should they be set?

Ideally, weights should be set such that the resulting model is highly likely to generate the training examples. In other words, the training examples---the patterns whose style we want to reproduce---should have high probability under the model. This weight-tuning problem is an instance of maximum-likelihood parameter estimation~\cite{PGMBook}. The log-likelihood function of our model is
%% log-likelihood function
\begin{equation*}
\ell(\weights : \dataset) =
	\sum_{(\pattern, \colors) \in \dataset}
	(
		\sum_{\term \in \model}
			\weights_\term \cdot \termStats(\colors, \pattern)
	)			
		- \ln{Z_\pattern(\weights)}
\end{equation*}
%%
whre $\dataset$ is the dataset of example pattern colorings. Convex log-likelihoods such as this one are typically maximized via gradient ascent. The partial derivatives of this function with respect to the weights are given by
%% Partial derivatives
\begin{equation*}
\frac{\partial}{\partial \weights_\term} \ell(\weights : \dataset) = 
	\sum_{(\pattern, \colors) \in \dataset}
			\termStats(\colors, \pattern)
		- \expectation_\weights[\termStats(\colorVars, \pattern)]
\end{equation*}
%%
where $\expectation_\weights$ denotes an expectation under the model with weights $\weights$. Unfortunately, these quantities are extremely expensive to compute. The expectation term requires probabilistic inference---an NP-complete problem---for every training pattern, for every iteration of gradient ascent.

This computational intractability has motivated the development of alternative, `biased' parameter estimation schemes which do not directly maximize the likelihood function but nevertheless yield parameters that give high likelihoods~\cite{NonMLEParameterEstimation}. We use one such method, called \emph{Contrastive Divergence} (CD)~\cite{ContrastiveDivergence}. CD uses the following approximation to the likelihood gradient:
%% CD gradient
\begin{equation*}
CD_k(\weights : \dataset) = 
	\sum_{(\pattern, \colors) \in \dataset}
			\termStats(\colors, \pattern)
		 -\termStats(\hat{\colors}, \pattern)
\end{equation*}
%%
where $\hat{\colors}$ is the coloring state obtained by running an MCMC chain for $k$ steps from the initial state $\colors$. CD essentially forms a local approximation to the likelihood gradient around the neighborhood of state $\colors$. Larger $k$ yields more accurate approximations at additional cost; we use $k = 10$ in our experiments. We initialize the weights uniformly to 1 and constrain them to be nonnegative, since all terms in the model are log-probabilities.

While the exact weights learned depend on the training dataset, there are several persistent trends.
The perceptual difference, color compatibility, and color name count terms receive the highest weights. These trends coincide well with our intuition that colors should be harmonious, adjacent regions should have sufficient contrast, and colors should be categorically similar to those in the training set.
The lowest weight belongs to the color name similarity term, which suggests that the similarity in how two colors are named is not strongly predictive of their compatibility as an adjacent color pair.

%Perceptual difference typically has the most weight, reflecting the intuitive notion that adjacent regions should have sufficient contrast.
%The terms for colorfulness and relative colorfulness also receive relatively high weight, which indicates that it is important for some regions to `pop' out of the background while others should remain muted. The color compatibility term has higher weight than several other terms, but not as high a weight as one might expect, given the seeming importance of overall color coordination. This might be explained by the fact that other terms redundantly encode some of the same properties that the O'Donovan color compatibilility model uses as predictive features. The lowest weight belongs to the name similarity term, which suggests that the similarity in how two colors are named is not strongly predictive of their compatibility as an adjacent color pair.

\subsection{Implementation}
\label{sec:implementation}

Our prototype implementation of this model is written in the Scala programming language, using the Factorie toolkit for probabilistic modeling~\cite{Factorie}. To evaluate the color compatibility term, it uses the reference MATLAB implementation provided by O'Donovan et al.~\shortcite{ODonovan}. A link to the source code can be found on the project website.
\section{Results}
\label{sec:results}

The data-driven probabilistic model we have presented produces good-looking patterns and can be applied in many different ways. The factor graph representation makes it easy to adapt our framework to handle new problems or incorporate user-provided design constraints by introducing additional factors to the factor graph, or changing the source data used to train the factor weights and histograms.

\remark{Discuss parameters used to train model ``unless otherwise specified''. Number of artists, weights, etc.}

\remark{Describe how we use the Colourlovers site to render the final images.}
The results shown below are rendered from the original vector pattern using the Colourlovers interface, from color assignments generated by our model. 

\subsection{Coloring Pattern Templates}

\paragraph{Automatic pattern coloring} In the most direct application of our framework, we can sample from our model to produce colorings for a pattern template that are similar to the colorings used for training. Figure~\ref{fig:teaser} shows two examples of this process. The sampled patterns exhibit a range of colors and styles employed by the Colourlover artists. For comparison, the same patterns colored with palettes randomly sampled from RGB-space are shown on the right. These patterns exhibit significant problems, such as low color harmony and adjacent regions with equi-limuinant colors.

\begin{figure*}[ht]
\begin{tabular}{ccc}
\includegraphics[width=.15\linewidth]{figs/permutationTemplatePalette} & \includegraphics[width=.4\linewidth]{figs/permutationBest8} & \includegraphics[width=.4\linewidth]{figs/permutationWorst8} %& \includegraphics[width=.12\linewidth]{figs/permutationArtist}
  \\
\textbf{(a)} Input Pattern & \textbf{(b)} Highest-scoring assignments & \textbf{(c)} Lowest-scoring assignments %& \textbf{(d)} Artist assignment
\\
\end{tabular}

\caption{Given a segmented image and corresponding palette as input, we use our color model to compute the likelihood of each possible assignment of the palette to the image regions. \textbf{(b)} and \textbf{(c)} show the top-eight and bottom-eight assignments. The assignment provided by the artist received the second-highest score and is highlighted in blue.}
\label{fig:permutation}
\end{figure*}

\paragraph{Coloring with fixed palettes} In some cases, an artist already knows what colors they want to use in an image. They might have found a palette they are enamored with elsewhere, or they might have a very specific theme in mind. Even with a fixed palette, there are still a range of images that can be created by mapping different colors to different regions, only some of which are desirable. To support this task, we use our model's score to rank all possible permutations of the colors. Figure~\ref{fig:permutation} shows one example of this process. \ref{fig:permutation}b shows the eight highest-rated color assignments which exhibit a variety of color styles, such as using four different background colors. On the other hand, the lowest-rated assignments all use the tangerine background color. Our color model assigns a very low score to using this color for the background region because its color properties are not similar to background colors in the training set. The actual color assignment originally provided by the artist for this color template received the second-highest score, suggesting that our model was able to capture the artist's intent.

\begin{figure*}[ht]
\begin{tabular}{cc} 
\includegraphics[width=.475\linewidth]{figs/constrainedSearchUnconstrained}&\includegraphics[width=.475\linewidth]{figs/constrainedSearchConstrained}\\
Unconstrained sampling&Constrained sampling\\
\end{tabular}

\caption{An artist coloring a pattern is presented with the results shown on the left, and decides that she only likes results where the stem of the plant is dark green. On the right, we use conditional inference to sample from our model subject to the constraint that the desired palette entry is fixed to a specific color. This is a natural way to incorporate semantic information about region colors which cannot be easily learned by our model.}
\label{fig:constrainedInference}
\vspace{-1.0em}
\end{figure*}

\remark{I realize that we may not keep in the hard or soft constraint figures and results, in favor of more ``interesting'' constraints; I'm still trying to find better examples of the hard and soft constraints though.}

\paragraph{Hard color constraints} Often a user has some idea of what types of palettes they are interested in. When this idea takes the form of a hard constraint on a region color, it is straightforward to accommodate this by using constrained inference to sample from the factor graph. Figure~\ref{fig:constrainedInference} shows an example. The model samples from the highest-scoring patterns subject to the constraint that the plant stem must be a specific shade of green.

\begin{figure}[ht]
\begin{tabular}{cc}
\raisebox{2em}{\includegraphics[width=.22\columnwidth]{figs/guidedSearch0Original}}&\includegraphics[width=.7\columnwidth]{figs/guidedSearch0MMR}\vspace{0.5em}\\
\raisebox{2em}{\includegraphics[width=.22\columnwidth]{figs/guidedSearch1Original}}&\includegraphics[width=.7\columnwidth]{figs/guidedSearch1MMR}\vspace{0.5em}\\
\raisebox{2em}{\includegraphics[width=.22\columnwidth]{figs/guidedSearch2Original}}&\includegraphics[width=.7\columnwidth]{figs/guidedSearch2MMR}\\
Suggestion&Results\\
\end{tabular}

\caption{An artist provides an initial color assignment and asks for patterns that are similar. We incorporate this request by adding an additional factor to our model, showing eight samples drawn from the new model for each of the suggested images.~\remark{D: It might also be cool to have a version of this that uses a photograph as the target...}}
\label{fig:nearbySuggestions}
\vspace{-1.0em}
\end{figure}

\paragraph{Soft color constraints} When the user has only a general idea of what kind of palette they want, soft constraints can be incorporated by adding new factors. A simple constraint is that the color assigned to a region should be close to a target color, which we represent using a factor of the form:

\begin{equation*}
\factor(\colorVars_\group | \pattern) = \mathcal{N}(||\colorVars_\group - \textrm{targetColor}(\group)||, \sigma_\textrm{user})
\end{equation*}

Here, $\textrm{targetColor}(\group)$ is the desired color of group $\group$ and $\sigma_\textrm{user}$ controls the extent to which the group is allowed to deviate from the desired color. We assign this factor a weight $w_\textrm{user} * w_\textrm{model}$, where $w_\textrm{model}$ is the sum of the weights of all other factors in the model. $w_\textrm{user}$ controls the tradeoff between satisfying the user-specified target colors for a region versus satisfying the color distribution encoded by the trained model.

This factor corresponds to adding a new term in our log-linear model with this statistics function:

\begin{equation*}
\termStats(\colors | \pattern) = \sum_{\group \in \groups}{\ln \mathcal{N}(||\colors_\group - \textrm{targetColor}(\group)||, \sigma_\textrm{user})}
\end{equation*}

Figure~\ref{fig:nearbySuggestions} shows examples where a proposed coloring is given, and the artist asks for colorings that are similar; this results in a new factor being added to each input color region. %\remark{Discussion on hold since the results will likely change}

\begin{figure}[ht]
\begin{tabular}{ccc} 
Style&Example&Results\\ %\hline
\raisebox{1.55em}{\emph{Light}}&\includegraphics[width=.148\columnwidth]{figs/styleResultsLightExample}&\includegraphics[width=.62\columnwidth]{figs/styleResultsLight}\vspace{0.5em}\\
\raisebox{1.55em}{\emph{Dark}}&\includegraphics[width=.148\columnwidth]{figs/styleResultsDarkExample}&\includegraphics[width=.62\columnwidth]{figs/styleResultsDark}\vspace{0.5em}\\
\raisebox{1.55em}{\emph{Bold}}&\includegraphics[width=.148\columnwidth]{figs/styleResultsBoldExample}&\includegraphics[width=.62\columnwidth]{figs/styleResultsBold}\vspace{0.5em}\\
\raisebox{1.55em}{\emph{Mellow}}&\includegraphics[width=.148\columnwidth]{figs/styleResultsMellowExample}&\includegraphics[width=.62\columnwidth]{figs/styleResultsMellow}\vspace{0.5em}\\
\end{tabular}

\caption{In this example, 17 patterns were chosen in three different styles, and a representative image from each style is shown in the second column. A separate model was then trained on each style, and in the third column we show four samples drawn from each model. In each case, our model is able to learn different properties of the desired distribution over colors.~\remark{Confusion matrix/matrices?}}
\label{fig:styleTraining}
\vspace{-1.0em}
\end{figure}

\begin{figure*}[ht]
\begin{tabular}{cc} 
\includegraphics[width=.48\linewidth]{figs/styleSugarExamples}&\includegraphics[width=.48\linewidth]{figs/styleAlbenajExamples}\vspace{1.0em}\\
\includegraphics[width=.48\linewidth]{figs/styleSugar}&\includegraphics[width=.48\linewidth]{figs/styleAlbenaj}\\
Artist A&Artist B\\
\end{tabular}

\caption{Our data-driven approach makes it easy to capture the styles of different artists. Top: representative images from two different artists. Bottom: results sampled from a model trained on 100 images from the artist.}
\vspace{-1.0em}
\label{fig:artistTraining}
\end{figure*}

\paragraph{Style capture} A powerful advantage of a data-driven approach is the ability to modify the underlying training source to achieve specialization of the resulting model. This allows us to capture a specific style and color preferences such as ``high-contrast palettes'' simply by selecting a set of images with the desired property. Figure~\ref{fig:styleTraining} demonstrates this using four style categories: \emph{Light}, \emph{Dark}, \emph{Bold}, and \emph{Mellow}. With only 17 training examples, our model is still able to capture general properties of the images such as the distribution of colors over the background regions in \emph{Light} and \emph{Dark}, and the amount of contrast between adjacent regions in \emph{Bold} and \emph{Mellow}. It is also easy to capture the style of a specific artist, as shown in Figure~\ref{fig:artistTraining}. Here, 100 images from each artist were used for training. The sampled images mimic certain properties of the the style of each artist, such as the light backgrounds preferred by artist A and the bold colors and dark backgrounds preferred by artist B.

\subsection{Applications}

\paragraph{Web design}

\paragraph{3D scene design}

\subsection{Performance}

\remark{Flesh this out with real data}
We measured the performance of our prototype implementation using wall-clock time on a machine with X specs~\remark{Use Matt's machine?}.

Table ? shows the time it takes to train the model for different numbers of training examples~\remark{Show it split into phases: cross-validation(?), regressor training, weight-tuning (we may want to try fewer iterations of weight-tuning w/ higher learning rate?}. Most of the time is taken up by X. Training amounts to a one-time-cost; once a model has been trained, it can suggest multiple colorings for many different patterns.

The time required to generate coloring suggestions varies depending on the visual complexity of the pattern, but on average takes X time in our experiments. Time breaks down into: running parallel tempering, running MMR (we retrieve the top 20 samples using MMR $\lambda = 0.5$).

There are many avenues for improving the performance of our unoptimized prototype. We could leverage the massive parallelism of graphics hardware to speed up parallel tempering, as was done in related work on automatic furniture layout~\cite{MerrellFurnitureLayout}. We could also improve the convergence of the individual MCMC chains via gradient-based proposals, as in Hamiltonian MCMC~\cite{HamiltonianMCMC}. Prior work has shown the feasibility of computing the gradients of programmatically-expressed distributions using automatic differentation~\cite{AutoDiff}. 
\section{Evaluation}
\label{sec:evaluation}

% Motivation for evaluation design
% May be more general aspects of aesthetics that our model captures, compared to random colorings or color compatibility only
We conducted a judgement study to better understand how coloring suggestions from our model compare to colorings from simpler models as well as colorings made by artists. We recruited 16 Computer Science graduate students (5 female) to participate in the study. All participants had normal color vision. To simulate the exploratory situations in which our model would be used, we generate and present multiple coloring suggestions. Although different participants almost certainly have different aesthetic preferences, we should see any general trends in the preferability of colorings generated by different methods. 

% Experiment Setup
\newcommand{\artistSource}{\emph{Artist}}
\newcommand{\modelSource}{\emph{Model}}
\newcommand{\compatSource}{\emph{Color Compatibility}}
\newcommand{\randomSource}{\emph{Random}}
We first generate coloring suggestions from four different sources----\artistSource~colorings, our \modelSource, a \compatSource-only model using the measure by O'Donovan et. al.~\shortcite{ODonovan}, and \randomSource~colorings---to be compared in the study. Next, we select 15 different pattern templates that do not have strong semantic associations and are also outside of our training set. For each pattern template, we then generate 4 coloring suggestions per source. For the \artistSource~source, we randomly choose 4 colorings from the top 45 artist colorings on COLOURlovers. For our \modelSource, we sample colors using parallel tempering for 2000 iterations and choose the top 4 results using MMR with $\lambda = 0.5$. We similarly sample \compatSource-only colorings. Finally, for the \randomSource~source, we generate 4 random colorings.

The study interface presents participants with a randomized grid of all 16 coloring suggestions for a pattern template. Participants are asked to chose the 4 colorings they like the most and the 4 they like the least from the grid before moving on to the next template. Each participant responds to 5 pattern templates randomly drawn from the pool of 15 and presented in random order. One template is presented twice to check for participant consistency.

Figure ~\ref{fig:study} shows the percentage of suggestions from each source that particpants chose as a `Top 4' or `Bottom 4' pattern. When counting, we do not include suggestions from the replicated template. We use binomial regression to model the odds of each source being chosen as either a `Top 4' or `Bottom 4' suggestion. Tukey all-pair comparison tests show that all differences are significant ($p < 0.01$), except for the odds that an \artistSource~suggestion will be chosen as a `Bottom 4' suggestion versus a \modelSource~suggestion. \artistSource~patterns are selected most often as a `Top 4' pattern, with our \modelSource~patterns selected second most often. In addition, \modelSource~patterns are selected least as a `Bottom 4' pattern, along with \artistSource~patterns.

\begin{figure}[h!]
  \begin{center}
  \includegraphics[width=\columnwidth]{figs/evaluation.png}
	\end{center}
\caption{The percentage of times that \artistSource-created, \modelSource-generated, \compatSource-only, and \randomSource~patterns were chosen by participants as one of their `Top 4' favorite or `Bottom 4' least favorite patterns in our online experiment.}

 \label{fig:study}
\end{figure}

These results suggest that \modelSource-generated patterns have significantly higher quality than patterns generated by automatic baselines. They do not yet achieve the same quality as \artistSource-created patterns, but since our model functions as part of an interactive coloring workflow, this result is acceptable. This research does not seek to replace human artists but rather to augment their abilities. The result that \modelSource~patterns are `disliked' as infrequently as \artistSource~patterns suggests that the \modelSource~patterns not chosen as favorites are likely good enough to serve as inspirational starting points for further creative work.
\section{Discussion and Future Work}
\label{sec:discussion}

Summarize, restate contributions.

Discuss...stuff.
Some suggestions:
\begin{itemize}
\item{How many training images are needed? (though maybe this should be part of the results section. We could look at the change in weight/coefficient differences as the number of training examples increases. Or possibly the fluctation in model score as number of training images increases?)}
\item{What the model does well and what doesn't it do well? (i.e. maybe it does better on certain types of patterns?) }
\end{itemize}


Some ideas for future work:
\begin{itemize}
 \item{Making it real-time: GPU implementation of parallel tempering (cite Merrell)}
 \item{Making it real-time: Hamiltonian MCMC via automatic differentiation (cite Wingate)}
 \item{Beyond a fixed number of color groups: Transdimensional inference.}
  \item{Working with vector artwork, instead of bitmaps...They are much cleaner. Vector/layered artwork is also interesting due to possible blending effects (like mentioned below). Though one good thing about using bitmaps is that they are more readily available.}
 \item{More sophisticated templates and/or model: `Soft' segments with `soft' membership in color groups? Useful for color blending, dealing with more sophisticated artwork.}
 \item{Templatizing images: In line with the above..automatically or semi-automatically generating pattern templates from images. In our training set, we're provided the source color palette, but for other images, we don't necessarily have that. Currently, we could extract a color theme from the image using K-means or some other technique, and then do a quantization, but this is not very sophisticated and doesn't work well for complicated patterns}
 \item{Semantics: use the web to associate color distributions with labels, so that a user can label a region as 'skin' and we can respect that}
\end{itemize}
%\section*{Acknowledgments}

Support for this research was provided by Intel (ISTC-VC) and an SAP Stanford Graduate Fellowship.
\remark{Other funding sources?}
We would also like to thank COLOURLovers \textit{jilbert} (Fig 1,2), \textit{symea} (Fig 1, 10), \textit{ArrayOfLilly} (Fig 3,5,7), \textit{COLOURLover} (Fig 6, 12), \textit{timanttimaari} and \textit{dazzlement} (Fig 7), \textit{Any Palacios} (Fig 8-12), \textit{gregreis} and \textit{praxicalidocious} (Fig 11), \textit{bhsav} and \textit{magg} (Fig 11, 12), \textit{vannea} and \textit{casslovescolors} (Fig 12), \textit{ivy21} (Fig 13), and \textit{caseycastille} (Fig 15) for their pattern templates; \textit{AlineDam} (Fig 9), and \textit{sugar!} and \textit{albenaj} (Fig 10) for the pattern coloring examples; and Flickr users \textit{marctasman} and \textit{zoomion} for the reference photographs (Fig 7).   


\bibliographystyle{acmsiggraph}
\bibliography{patternColoring}

\section*{Appendix}
\label{sec:appendix}

Here we provide a complete listing of features and precise definitions for some of the color properties used in our model.

\subsection*{Color Properties}

\begin{description}
	
\item[Binary] \hfill
	\begin{description}[leftmargin=*]
	  \item[Chroma Difference] measures the squared fraction of perceptual distance due to the chroma channels: $\frac{\delta a^2+\delta b^2}{\delta a^2+\delta b^2+\delta L^2}$
	\end{description}
	
\end{description}

\subsection*{Features}

\begin{description}

\item[Adjacency Features] \hfill
	\begin{description}[leftmargin=*]
	  \item[Enclosure Strengths] are two values which measure how much one neighbor in the adjacency encloses the other and vice versa. Enclosure strength is defined as the number of pixels of the neighboring segment appearing within a 2-pixel neighborhood outside the segment's boundary, normalized by the area of that neighborhood. Out-of-image pixels are counted as part of the neighborhood area.
	  \item[Unary Segment Features] The adjacency feature set also includes the segment feature vectors of the participating segments. We concatenate the two vectors, putting the one with the smallest $L_2$ norm first to enforce a consistent ordering.
	\end{description}
	
\end{description}

\end{document}
